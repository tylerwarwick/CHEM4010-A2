% Gemini theme
% https://github.com/anishathalye/gemini

\documentclass[final]{beamer}

% ====================
% Packages
% ====================

\usepackage[T1]{fontenc}
\usepackage{lmodern}
% Come back and adjust to height of 72
\usepackage[size=custom,width=120,height=120,scale=1.0]{beamerposter}
\usetheme{gemini}
\usecolortheme{rpi}
\usepackage{graphicx}
\usepackage{booktabs}
\usepackage{tikz}
\usepackage{pgfplots}
\pgfplotsset{compat=1.14}
\usepackage{anyfontsize}
\usepackage{comment}
%\usepackage{wrapfig}
\usepackage{caption}  % needed for \captionof outside floats
\usepackage[super]{natbib}  % superscript numbers like ACS
\bibliographystyle{achemso}  % ACS-approved style

% ====================
% Lengths
% ====================

% If you have N columns, choose \sepwidth and \colwidth such that
% (N+1)*\sepwidth + N*\colwidth = \paperwidth
\newlength{\sepwidth}
\newlength{\colwidth}
\setlength{\sepwidth}{0.025\paperwidth}
\setlength{\colwidth}{0.3\paperwidth}

\newcommand{\separatorcolumn}{\begin{column}{\sepwidth}\end{column}}
%\newcommand{\separatorcolumn}{\column{\sepwidth}}


% Wrapfig padding
%\setlength{\intextsep}{1em}   % vertical space above & below the figure
\setlength{\columnsep}{1em}   % horizontal space between figure & text

% ====================
% Title
% ====================

\title{Ketamine: Stereospecific Synthesis Required}

\author{Tyler Warwick (1130124)}

%\institute[shortinst]{\inst{1} Some Institute \samelineand \inst{2} Another Institute}

% ====================
% Footer (optional)
% ====================

\begin{comment}
    

\footercontent{
  \href{https://www.example.com}{https://www.example.com} \hfill
  ABC Conference 2025, New York --- XYZ-1234 \hfill
  \href{mailto:alyssa.p.hacker@example.com}{alyssa.p.hacker@example.com}}
% (can be left out to remove footer)

\end{comment}
% ====================
% Logo (optional)
% ====================

% use this to include logos on the left and/or right side of the header:
%\logoright{\includegraphics[height=7cm]{logo1.pdf}}
%\logoleft{\includegraphics[height=7cm]{logo2.pdf}}

% Blocks I am retaining
\begin{comment}
  \begin{alertblock}{A highlighted block}

    This block catches your eye, so \textbf{important stuff} should probably go
    here.

    Curabitur eu libero vehicula, cursus est fringilla, luctus est. Morbi
    consectetur mauris quam, at finibus elit auctor ac. Aliquam erat volutpat.
    Aenean at nisl ut ex ullamcorper eleifend et eu augue. Aenean quis velit
    tristique odio convallis ultrices a ac odio.

    \begin{itemize}
      \item \textbf{Fusce dapibus tellus} vel tellus semper finibus. In
        consequat, nibh sed mattis luctus, augue diam fermentum lectus.
      \item \textbf{In euismod erat metus} non ex. Vestibulum luctus augue in
        mi condimentum, at sollicitudin lorem viverra.
      \item \textbf{Suspendisse vulputate} mauris vel placerat consectetur.
        Mauris semper, purus ac hendrerit molestie, elit mi dignissim odio, in
        suscipit felis sapien vel ex.
    \end{itemize}

    Aenean tincidunt risus eros, at gravida lorem sagittis vel. Vestibulum ante
    ipsum primis in faucibus orci luctus et ultrices posuere cubilia Curae.

  \end{alertblock}
  \begin{block}{A block containing an enumerated list}

    Vivamus congue volutpat elit non semper. Praesent molestie nec erat ac
    interdum. In quis suscipit erat. \textbf{Phasellus mauris felis, molestie
    ac pharetra quis}, tempus nec ante. Donec finibus ante vel purus mollis
    fermentum. Sed felis mi, pharetra eget nibh a, feugiat eleifend dolor. Nam
    mollis condimentum purus quis sodales. Nullam eu felis eu nulla eleifend
    bibendum nec eu lorem. Vivamus felis velit, volutpat ut facilisis ac,
    commodo in metus.

    \begin{enumerate}
      \item \textbf{Morbi mauris purus}, egestas at vehicula et, convallis
        accumsan orci. Orci varius natoque penatibus et magnis dis parturient
        montes, nascetur ridiculus mus.
      \item \textbf{Cras vehicula blandit urna ut maximus}. Aliquam blandit nec
        massa ac sollicitudin. Curabitur cursus, metus nec imperdiet bibendum,
        velit lectus faucibus dolor, quis gravida metus mauris gravida turpis.
      \item \textbf{Vestibulum et massa diam}. Phasellus fermentum augue non
        nulla accumsan, non rhoncus lectus condimentum.
    \end{enumerate}

  \end{block}

  \begin{block}{Fusce aliquam magna velit}

    Et rutrum ex euismod vel. Pellentesque ultricies, velit in fermentum
    vestibulum, lectus nisi pretium nibh, sit amet aliquam lectus augue vel
    velit. Suspendisse rhoncus massa porttitor augue feugiat molestie. Sed
    molestie ut orci nec malesuada. Sed ultricies feugiat est fringilla
    posuere.

    \begin{figure}
      \centering
        \begin{minipage}{.5\textwidth}
            \centering
            \includegraphics[width=.7\linewidth]{example-image-a}
            \caption{A subfigure}
        \end{minipage}%
        \begin{minipage}{.5\textwidth}
            \centering
            \includegraphics[width=.7\linewidth]{example-image-b}
            \caption{Another subfigure}
        \end{minipage}
      \caption{Example of a figure side by side.}
    \end{figure}

  \end{block}

  \begin{block}{Nam cursus consequat egestas}

    Nulla eget sem quam. Ut aliquam volutpat nisi vestibulum convallis. Nunc a
    lectus et eros facilisis hendrerit eu non urna. Interdum et malesuada fames
    ac ante \textit{ipsum primis} in faucibus. Etiam sit amet velit eget sem
    euismod tristique. Praesent enim erat, porta vel mattis sed, pharetra sed
    ipsum. Morbi commodo condimentum massa, \textit{tempus venenatis} massa
    hendrerit quis. Maecenas sed porta est. Praesent mollis interdum lectus,
    sit amet sollicitudin risus tincidunt non.

    \begin{itemize}
      \item \textbf{Sed consequat} id ante vel efficitur. Praesent congue massa
        sed est scelerisque, elementum mollis augue iaculis.
        \begin{itemize}
          \item In sed est finibus, vulputate
            nunc gravida, pulvinar lorem. In maximus nunc dolor, sed auctor eros
            porttitor quis.
          \item Fusce ornare dignissim nisi. Nam sit amet risus vel lacus
            tempor tincidunt eu a arcu.
          \item Donec rhoncus vestibulum erat, quis aliquam leo
            gravida egestas.
        \end{itemize}
      \item \textbf{Sed luctus, elit sit amet} dictum maximus, diam dolor
        faucibus purus, sed lobortis justo erat id turpis.
      \item \textbf{Pellentesque facilisis dolor in leo} bibendum congue.
        Maecenas congue finibus justo, vitae eleifend urna facilisis at.
    \end{itemize}

  \end{block}
\end{comment}

% ====================
% Body
% ====================

\begin{document}

% Refer to https://github.com/k4rtik/uchicago-poster
% logo: https://www.cam.ac.uk/brand-resources/about-the-logo/logo-downloads

\begin{comment}
\addtobeamertemplate{headline}{}
{
    \begin{tikzpicture}[remember picture,overlay]
      \node [anchor=north west, inner sep=3cm] at ([xshift=0.0cm,yshift=3.0cm]current page.north west) {};
     %{\includegraphics[height=6.5cm]{logo/Rensselaer Large Logo CMYK-White.png}};
      
    \end{tikzpicture}
}
\end{comment}

\begin{comment}
  Central image block (good for use later with enantiomer)
  \begin{block}{A block title}

    Some block contents, followed by a diagram, followed by a dummy paragraph.

    \begin{figure}
      \centering
      \begin{tikzpicture}[scale=6]
        \draw[step=0.25cm,color=gray] (-1,-1) grid (1,1);
        \draw (1,0) -- (0.2,0.2) -- (0,1) -- (-0.2,0.2) -- (-1,0) -- (-0.2,-0.2) -- (0,-1) -- (0.2,-0.2) -- cycle;
      \end{tikzpicture}
      \caption{A figure caption.}
    \end{figure}

    Lorem ipsum dolor sit amet, consectetur adipiscing elit. Morbi ultricies
    eget libero ac ullamcorper. Integer et euismod ante. Aenean vestibulum
    lobortis augue, ut lobortis turpis rhoncus sed. Proin feugiat nibh a
    lacinia dignissim. Proin scelerisque, risus eget tempor fermentum, ex
    turpis condimentum urna, quis malesuada sapien arcu eu purus.

  \end{block}

  WRAPFIG OPTION - sketchy
   \begin{wrapfigure}{r}{0pt} % width=0pt since we’ll control height
      \centering
      \includegraphics[height=0.20\linewidth]{example-image} % replace with your graphic
      \caption{A figure caption.}
    \end{wrapfigure}

    Lorem ipsum dolor sit amet, consectetur adipiscing elit. Morbi ultricies
    eget libero ac ullamcorper. Integer et euismod ante. Aenean vestibulum
    lobortis augue, ut lobortis turpis rhoncus sed. Proin feugiat nibh a
    lacinia dignissim. Proin scelerisque, risus eget tempor fermentum, ex
    turpis condimentum urna, quis malesuada sapien arcu eu purus.

    Curabitur vitae velit nec sapien malesuada tincidunt. Suspendisse potenti.
    Donec nec risus et urna eleifend ultricies. Duis a metus eget lorem
    pellentesque fermentum.

\end{comment}


\begin{frame}[t]
\begin{columns}[t]
\separatorcolumn{

\begin{column}{\colwidth}

  \begin{block}{A Brief History}
    Ketamine (nicknamed “Special K”) is a derivative of a class of compounds called arylcyclohexylamines. 
    These compounds are psychoactive and are most notable for their strong dissociative effects\cite{pelixo_2024_determination}. 
    Development of the arylcyclohexylamine class began with the synthesis of phencyclidine (PCP) in 1956, where it showed 
    promise as an anesthetic. However, strong psychological side effects disqualified it from being clinically viable. Ketamine 
    was developed in 1962 in hopes of retaining PCP's abilities as a CNS depressant without the harsh side effects\cite{ivy_2012_designer}. 

    \begin{center}
      \includegraphics[width=0.6\linewidth]{media/aryl.jpg}
      %\includegraphics[width=0.7\linewidth]{media/ket-enantiomers.jpg} 
      \captionof{figure}{Ketamine structure.\cite{pelixo_2024_determination}} 
    \end{center}
    \vspace{0.5pt}
   
    In the early 2000s, ketamine began to be explored within psychiatry as a potential antidepressant, particularly 
    for treatment-resistant depression (TRD) (CITATION OF SOME SORT). Ketamine is a chiral molecule and has historically 
    been administered as a racemic mixture containing equal parts R- and S- enantiomers, also called arketamine and esketamine, 
    respectively. Only within the last decade has research begun to distinguish the unique pharmacological profiles of the 
    two enantiomers\cite{jelenlukea_2020_ketamine}. 

    The most recent update came in 2019, when Johnson and Johnson received FDA approval for an esketamine nasal spray called 
    Spravato\textsuperscript{\textregistered}\cite{nationalinstituteofhealth_2019_ketamine}, which  remains the only FDA-approved ketamine intervention that can be 
    used in a psychiatric context. Racemic ketamine continues to be utilized as an anesthetic but has  also begun being used 
    off label to treat TRD\cite{usfoodanddrugadministration_2022_potential}. However, ongoing research suggests enantiomerically pure 
    forms and their downstream metabollites may offer distinct therapeutic advantages. Consequently, racemic ketamine and its individual enantiomers are treated 
    as separate pharmaceutical entities, which has necessitated the development of distinct synthesis and manufacturing 
    processes\cite{jelenlukea_2020_ketamine}\cite{chen_2019_enantioselective}. 
  \end{block}

\end{column}
}
\separatorcolumn{

\begin{column}{\colwidth}
  \begin{block}{Synthesis}

    \begin{center}
      \includegraphics[width=0.9\linewidth]{media/racemic-synthesis.jpg}  
      %\includegraphics[width=0.7\linewidth]{media/ket-enantiomers.jpg}
      \captionof{figure}{Historical and modern synthetic strategies for racemic ketamine.\cite{yen_2023_a}}
    \end{center}

    Nam vulputate nunc felis, non condimentum lacus porta ultrices. Nullam sed
    sagittis metus. Etiam consectetur gravida urna quis suscipit.

    \begin{center}
      \includegraphics[width=0.9\linewidth]{media/recycling-ket.jpeg}
      \captionof{figure}{Recycling / stereospecific synthesis\cite{gao_2020_process}}
    \end{center}
    

  \end{block}
    
  
\end{column}
}

\separatorcolumn{

\begin{column}{\colwidth}

  \begin{exampleblock}{A highlighted block containing some math}

    A different kind of highlighted block.
    
    Interdum et malesuada fames $\{1, 4, 9, \ldots\}$ ac ante ipsum primis in
    faucibus. Cras eleifend dolor eu nulla suscipit suscipit. Sed lobortis non
    felis id vulputate.

    \heading{A heading inside a block}

    Praesent consectetur mi $x^2 + y^2$ metus, nec vestibulum justo viverra
    nec. Proin eget nulla pretium, egestas magna aliquam, mollis neque. Vivamus
    dictum $\mathbf{u}^\intercal\mathbf{v}$ sagittis odio, vel porta erat
    congue sed. Maecenas ut dolor quis arcu auctor porttitor.

    \heading{Another heading inside a block}

    Sed augue erat, scelerisque a purus ultricies, placerat porttitor neque.
    Donec $P(y \mid x)$ fermentum consectetur $\nabla_x P(y \mid x)$ sapien
    sagittis egestas. Duis eget leo euismod nunc viverra imperdiet nec id
    justo.

  \end{exampleblock}

  \begin{block}{Nullam vel erat at velit convallis laoreet}

    Class aptent taciti sociosqu ad litora torquent per conubia nostra, per
    inceptos himenaeos. Phasellus libero enim, gravida sed erat sit amet,
    scelerisque congue diam. Fusce dapibus dui ut augue pulvinar iaculis.

    \begin{table}
      \centering
      \begin{tabular}{l r r c}
        \toprule
        \textbf{First column} & \textbf{Second column} & \textbf{Third column} & \textbf{Fourth} \\
        \midrule
        Foo & 13.37 & 384,394 & $\alpha$ \\
        Bar & 2.17 & 1,392 & $\beta$ \\
        Baz & 3.14 & 83,742 & $\delta$ \\
        Qux & 7.59 & 974 & $\gamma$ \\
        \bottomrule
      \end{tabular}
      \caption{A table caption.}
    \end{table}

    Donec quis posuere ligula. Nunc feugiat elit a mi malesuada consequat. Sed
    imperdiet augue ac nibh aliquet tristique. Aenean eu tortor vulputate,
    eleifend lorem in, dictum urna. Proin auctor ante in augue tincidunt
    tempr. Proin pellentesque vulputate odio, ac gravida nulla posuere
    efficitur. Aenean at velit vel dolor blandit molestie. Mauris laoreet
    commodo quam, non luctus nibh ullamcorper in. Class aptent taciti sociosqu
    ad litora torquent per conubia nostra, per inceptos himenaeos.

    Nulla varius finibus volutpat. Mauris molestie lorem tincidunt, iaculis
    libero at, gravida ante. Phasellus at felis eu neque suscipit suscipit.
    Integer ullamcorper, dui nec pretium ornare, urna dolor consequat libero,
    in feugat elit lorem euismod lacus. Pellentesque sit amet dolor mollis,
    auctor urna non, tempus sem.

  \end{block}

  \begin{block}{References}
    %\nocite{*}
    \footnotesize
    \bibliography{References}
  \end{block}

\end{column}
}
%\separatorcolumn
\end{columns}
\end{frame}

\end{document}
